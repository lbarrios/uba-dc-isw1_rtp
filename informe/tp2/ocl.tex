\subsection{OCL}

\begin{listocl}

%
% - OCL: 1
%
\begin{itemocl}{La redituabilidad de una unidad es su precio de venta menos su precio de lista.}
Context: Unidad
Inv: self.redituabilidad = self.precioDeVenta - self.precioDeLista
  \end{itemocl}


%
% - OCL: 2
%
  \begin{itemocl}{La redituabilidad de un pedido de cliente es la ganancia menos el costo.}
Context: Pedido de Cliente

ganancia(p) = if p.estadoPedido == Concretado
	   then p.contiene->collect(redituabilidad)->sum()
	   else 0
	   endif
costo(p) = if p.seEnvio->notEmpty()
	then p.seEnvio.costo
	else 0
	endif
	+ if p.seEnvio->notEmpty() and p.seEnvio.oclIsTypeOf(EnvioFallido)
	then p.seEnvio.costoMercaderiaMalograda
	else 0
	endif
Inv: self.redituabilidad = ganancia(self) - costo(self)
  \end{itemocl}


%
% - OCL: 3
%
  \begin{itemocl}{Se le permite la contraentrega si y solo si la redituabilidad del cliente es mayor al umbral de redituabilidad.}
Context: Cliente Particular

redituabilidadCliente(c) = c.realizo->collect(redituabilidad)->sum()
Inv: self.permitidoContraentrega = redituabilidadCliente(self) > GLOBAL_VARIABLE(umbralRedituabilidad)
  \end{itemocl}


%
% - OCL: 4
%
  \begin{itemocl}{Un cliente particular no puede tener dos pedidos vigentes.}
Context: Cliente Particular
Inv: self.realizo->filter(x | {entregado, anulado, cancelado}->excludes(x.estado)).size() < 2
  \end{itemocl}

%
% - OCL: 5
%
  \begin{itemocl}{Un Producto está bajo en stock en un Depósito si y sólo sí la cantidad de unidades de dicho producto en el Depósito está por debajo del umbral mínimo de dicho producto.}
Context: Producto
Inv: Deposito.allInstances()->forall(deposito |
    (self.estaBajoEnStock->filter(deposito2 | deposito = deposito2).size() = 1) =
    (deposito.contiene->filter(unidad | unidad.correspondeA = self).size() < self.umbralMinimo)
)
  \end{itemocl}

%
% - OCL: 6
%
  \begin{itemocl}{Para cada pedido de cliente, $fechaCreacion < fechaDespacho < fechaEntrega$}
Context: Envio
Inv: self.correspondeA.fechaCreacion < self.fechaDespacho < self.correspondeA.fechaEntrega
  \end{itemocl}

%
% - OCL: 7
%
  \begin{itemocl}{
      Una unidad está en solo uno de los siguientes lugares:
          pedidoAProveedor
          depósito
          depósito y en pedido de cliente en estado enCarrito, anulado, cancelado
          en pedido de cliente confirmado
    }
Context: Unidad
Inv:
(self.esContenida->notEmpty() and self.almacenadaEn->isEmpty() and self.perteneceA->isEmpty())
or
(self.almacenadaEn->notEmpty() and self.esContenida->isEmpty() and self.perteneceA->isEmpty())
or
(self.almacenadaEn->notEmpty() and self.esContenida->isEmpty() and self.perteneceA->notEmpty() and {enCarrito, cancelado, anulado}->includes(self.perteneceA.estado))
or
(self.almacenadaEn->isEmpty() and self.esContenida->isEmpty() and self.perteneceA->filter(p | {enCarrito, cancelado, anulado}->excludes(p.estado)).size() == 1)
  \end{itemocl}

%
% - OCL: 8
%
  \begin{itemocl}{
      Un cliente particular C tiene un producto recomendado P si y solo si:
          C nunca compro P y
          P pertenece a la lista de 10 más comprados de algún cliente particular parecido a C

          Observación: la definición de \texttt{productosComprados} utiliza una navegación a través de conjuntos que se van aplanando automáticamente\footnote{http://st.inf.tu-dresden.de/files/general/OCLByExampleLecture.pdf}.
    }
  Context: Cliente Particular
  edad(cliente) = (currentTime() - cliente.nacimiento).years

  edadParecida(cliente1, cliente2) = |edad(cliente1) - edad(cliente2)| < 4

  productosComprados(cliente) = cliente.realizo.contiene.correspondeA

  cantUnidadesCompradas(producto, cliente) = productosComprados(cliente)->filter(p | p = producto)->size()

  cantUnidadesCompradasOrdenadas(cliente) = productosComprados(cliente)->asSet()->sortedBy(p | cantUnidadesCompradas(p, cliente))->reverse()

  diezMasComprados(cliente) = cantUnidadesCompradasOrdenadas(cliente)->subSequence(1, 10)

  clienteParecido(cliente1, cliente2) = edadParecida(cliente1, cliente2) and intersection(diezMasComprados(cliente1), diezMasComprados(c2))->size() >= 5

  Inv: self.seLeRecomiendan->forAll(producto | productosComprados(self)->excludes(producto)
      and ClienteParticular.allInstances()->exists(c | clienteParecido(self, c) and diezMasComprados(c)->includes(p))
  )
  \end{itemocl}

%
% - OCL: 9
%
  \begin{itemocl}{Comprobantes deben tener asociados datos de pago registrados por el cliente.
Observación: la tarjeta puede tener como titular a un familiar, pero quien la registra es el cliente.}
    Context: Comprobante de tarjeta de credito
    Inv: self.datosDePago.asociadoA = self.esPagoDe.fueRealizadoPor
    Context: Comprobante PayPal
    Inv: self.datosDePago.asociadoA = self.esPagoDe.fueRealizadoPor
  \end{itemocl}
\end{listocl}

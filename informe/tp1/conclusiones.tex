%--
%  Conclusiones: mencionar brevemente de qué formas les resultó más sencillo
%  encarar el TP. Por ejemplo en qué orden realizaron los diagramas, qué
%  aspectos presentaron las mayores dificultades, etc.
%--

\section{Conclusiones TP1}

El presente trabajo resultó una efectiva introducción a varios de los problemas
que la \texttt{Ingeniería de Software} busca resolver.

El primer paso fue plantear los \texttt{fenómenos} que consideramos forman parte
de la modificación a la cadena \textbf{Mes\%}. Empezaron a surgir así los
distintos conceptos que luego se transformarían en \texttt{agentes}, junto con
las primeras \texttt{presunciones de dominio}.

Luego, nos abocamos al desarrollo de un \texttt{modelo de objetivos}. Durante el
transcurso de este proceso fue necesario \texttt{refinar} nuestros datos sobre
varios asuntos por lo que hubo una \texttt{interacción intensa} con el tutor,
que hacía las veces de CEO de la cadena. El resultado de esta etapa de modelado
quedó plasmado en un \texttt{diagrama de objetivos}.

Una vez que el diagrama de objetivos estuvo finalizado, esbozar el
\texttt{diagrama de contexto} y ejemplificar los potenciales \texttt{escenarios
de uso} fue relativamente sencillo.

Algo que llamó la atención con respecto a trabajos prácticos de otras materias
fue una necesidad abordar los problemas de forma grupal y tomar decisiones
consensuadas. Todas las secciones del trabajo están conectadas y requirieron
coordinación activa.

\emph{Al hacer una evaluación sobre las dificultades que nos fueron surgiendo a
lo largo del desarrollo de este TP, nos encontramos con muchos puntos que valen
la pena destacar}.

Por un lado, la ya mencionada necesidad de mantener una comunicación fluída,
permanente, y clara entre todos los integrantes, junto con la dificultad de
coordinar las tareas, nos expuso a frecuentes confrontaciones y desacuerdos
(necesarios), que enlentecieron el desarrollo del trabajo.

Un vicio constante fue pensar en la implementación al definir estos modelos más
abstractos, limitando y forzando de forma prematura, quizás, el resultado.

La parte del mundo que estábamos describiendo nos resultó demasiado compleja
como para ser descripta fielmente a través de los modelos con los que
contábamos. Además de las limitaciones que cada uno de ellos mostró por
separado, al integrarlos se sintió la falta de herramientas que nos permitieran
modelar otros aspectos que, se presume, se explorarán más avanzada la materia.

Existió ciertamente una dificultad implementativa, relacionada directamente a la
complejidad de los diagramas necesarios, frente a la sencillez de las
herramientas utilizadas para realizarlos. Consideramos que esto es un detalle
que vale la pena mencionar, ya que la diferencia entre los software de diagramas
pagos (por ejemplo, Microsoft Visio) frente a aquellos gratuitos (por ejemplo,
draw.io) es \textbf{muy significativa}, lo que nos sumó una dificultad
innecesaria y que se encuentra más allá de la correcta aplicación de las
técnicas aprendidas.
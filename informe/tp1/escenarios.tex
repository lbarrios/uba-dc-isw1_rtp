%--
%-- Escenarios de uso
%--
\newpage
\subsection{Escenarios representativos de uso}

%-- Escenario 1
\subsubsection{A: Registro online}

\textbf{\emph{Alice}} no es cliente habitual de la cadena mesporciento, ya que
le queda lejos de su casa como para ir caminando. Tras enterarse de la nueva
posibilidad de comprar online, se conecta al sitio web www.mesporciento.com a
través de su computadora, y registra un nuevo usuario
``\texttt{alice\_gatita93}'', ingresando para ello sus \textbf{datos
personales}, tales como nombre, domicilio, teléfono, email, y una
\textbf{contraseña segura}. El sitio verifica que los datos ingresados son
correctos, y luego le envía un mail de confirmación con un vínculo en el que
\textbf{\emph{Alice}} presiona para validar su cuenta. \textbf{\emph{Alice}}
entonces reingresa al sitio, utilizando ahora su nuevo usuario.

%-- Escenario 2
\subsubsection{B: Registro presencial con sistema de reputación}

\textbf{\emph{Bob}} es cliente habitual de la cadena mesporciento, y aprovecha
una de sus rutinarias compras para registrar su usuario en su sitio web. Para
ello, se encargó previamente de juntar la documentación requerida para el
registro: el documento de identidad, y una acreditación de domicilio, en
particular lleva la última factura de luz. Luego de las compras, le pregunta a
la cajera en dónde debe registrarse, y le indican que se dirija a la sección de
informes.

En la sección de informes no hay nadie, por lo que \textbf{\emph{Bob}} debe
esperar unos 10 minutos hasta que aparezca la encargada. Esta le pide la
documentación, y la verifica mientras le entrega a \textbf{\emph{Bob}} un
cuestionario de datos personales para que lo complete. Luego,
\textbf{\emph{Bob}} le entrega el cuestionario, y la empleada verifica que los
datos del cuestionario y la documentación coincidan. Entonces, le saca
fotocopias a la documentación y al cuestionario. Le entrega la fotocopia del
cuestionario a \textbf{\emph{Bob}}, mientras que la fotocopia de la
documentación es archivada junto con el cuestionario en un sobre de papel
madera, que a su vez es apilado junto a muchos otros sobres de aspecto similar.
Finalmente, le informa a \textbf{\emph{Bob}} que se le avisará por mail en
cuanto el registro se encuentre completo, y allí mismo le brindarán las
instrucciones para acceder al sitio.

Luego de dos semanas, \textbf{\emph{Bob}} recibe un email de parte del remitente
felicitaciones@mesporciento.com, y asunto <<Bienvenido a una nueva forma de
comprar>>, en el que se le informa que ya se encuentra habilitado su usuario
``\texttt{marley\_b420}''

%-- Escenario 3
\subsubsection{C: Pago online}

\textbf{\emph{Charlie}} se conecta a la web de mesporciento desde su tablet, con
su usuario ``\texttt{je\_suis\_moi}'', dispuesto a iniciar su compra de
productos semanal, de forma online. Para ello, revisa el listado de productos, y
agrega a su \textbf{carrito virtual} los que necesita. Una vez que el carrito
contiene todos los productos que desea, presiona el botón \texttt{``FINALIZAR
COMPRA''}, el cual lo dirige a la pantalla de cierre de pedido. En esta
pantalla, se le informa del costo total de la compra, y se le ofrecen opciones
de fechas y horarios posibles de entrega, de entre las que
\textbf{\emph{Charlie}} elige el Miércoles de la semana que viene, por la
mañana, ya que sabe que en ese horario va a estar en su casa.

Luego, en la siguiente pantalla, elige la opción de Pago Online, y el sitio le
solicita que elija un método de pago de entre las distintas opciones
disponibles. Ya que \textbf{\emph{Charlie}} confía mucho en el sistema
\texttt{PayPal} (como ejemplo de Agente de Cobro), lo elige, tras lo cual se abre una ventana externa que redirige
al sitio de PayPal, en que tras ingresar su usuario y su clave se le solicita
confirmar el valor de la compra. Luego de que esta ventana se cierra, el sitio
le muestra una animación muy jocosa de un tigre mirando un reloj, mientras
debajo se puede leer la frase \texttt{``Por favor, espere, estamos validando el
pago...''}. Tras unos segundos, el tigre comienza a bailar, el mensaje se
desvanece, y lo reemplaza un nuevo mensaje \texttt{``Su pago se encuentra
confirmado. Le hemos enviado un mail con la información de su pedido. Gracias
por confiar en nosotros. En unos instantes, será redirigido a la página
principal.''}.

%-- Escenario 4
\subsubsection{D: Pago contrareembolso}

\textbf{\emph{Dave}} tiene un problema de adicción al casino. Normalmente, con
la ayuda de sus amigos y familiares, lo controla sin mayores inconvenientes.
Pero hace 1 semana tuvo un viaje laboral, y en el último día, libre para todos
los empleados, no resistió la tentación de jugar una o dos tiradas de ruleta,
con el efectivo que llevaba encima. Tuvo la mala suerte de que le fue
relativamente bien, ganó ambas jugadas lo que le hizo cuadruplicar su efectivo.
Envalentonado por su repentino y misterioso golpe de suerte, se dirigió a la
casilla de venta de fichas, y gastó todo el dinero de su cuenta bancaria en
fichas. También compró fichas con su tarjeta de crédito, en un pago, hasta
alcanzar el límite. Compró en total 450 fichas, y volvió a dirigirse a la
ruleta. Su plan inicial era realizar una paciente Martingala, pero un rayo
cósmico atravesó su mente momentos antes de colocar la apuesta, y supo entonces
que debía elegir el número 7. Claro, porque este era el séptimo día del viaje
laboral, y había tenido mucha suerte, por lo que el siete era un buen número.
Claramente, \textbf{\emph{Dave}} perdió todo su dinero, y no solo eso, sino que
se endeudó gravemente, saturando el límite de su tarjeta de crédito.

Al volver a su casa, le cuenta lo sucedido a su tío, pidiéndole que no le cuente
a nadie, y este le presta dinero ``hasta que logre salir de la situación''. Como
no tiene tiempo de ir al supermercado, ya que debe hacer horas extras para pagar
sus deudas, aprovecha el sistema de compras online de mesporciento para encargar
las provisiones de la semana durante la noche. Se autentica en el sitio con su
usuario, ``\texttt{lucky\_guy\_00}'', elige los productos indispensables para el
resto del mes, y pacta una fecha de entrega para el día siguiente. Al momento de
elegir la opción de pago, advierte que no puede realizar un pago online, ya que
la tarjeta se encuentra saturada, por lo que opta por elegir la opción de pago
contrareembolso.

Al otro día, temprano, suena el timbre, y recibe el pedido, el cual paga en
efectivo, y les deja una modesta propina a los muchachos para que carguen las
bolsas hasta la cocina de su casa.

%-- Escenario 5
\subsubsection{E: Entrega correcta}

El Martes 13 de Abril \textbf{\emph{Erin}} realizó un pedido de torta de
cumpleaños, cotillón y un regalo grandioso para ser entregado el Martes
siguiente, en conmemoración del 50-cumpleaños de su tía. Aprovecha la
posibilidad para elegir que su pedido sea entregado en la casa de su tía, y no
en su domicilio. Para ello, intenta modificar el domicilio de su usuario
``\texttt{ireland\_green}'', pero el sistema no se lo permite. Entonces, como
\textbf{\emph{Erin}} es muy inteligente, crea un nuevo usuario
``\texttt{tia\_50}'', especial para esta ocasión, el cual completa con los datos
de su tía. Por comodidad, además, lo paga en línea, con una tarjeta de crédito
(la suya, no la de su tía), ya que por costumbre familiar está prohibido hablar
de dinero durante el cumpleaños de una tía, y quiere evitar malos momentos
durante el episodio festivo.

Llegado el día, están todos festejando, ya con algunas copitas encima, cuando la
tía grita a \textbf{\emph{Erin}}: <<¿y la torta? ¿y los juguetitos que me habías
prometido?>>. En ese instante, justamente, suena el timbre, y resultan ser los
empleados de mesporciento. \textbf{\emph{Erin}} les abre la puerta y les indica
dónde dejar la mercadería. Les dice que no puede darles propina por una
costumbre familiar, tras lo cual los empleados regresan a su camión,
apesadumbrados.

%-- Escenario 6
\subsubsection{F: Ausente durante entrega con límite de entregas fallidas fijo}

\textbf{\emph{Frank}} es una persona muy olvidadiza. Tanto es así, que durante
la mañana de hoy, fue hasta el banco a cobrar un cheque, para terminar dándose
cuenta que no lo había llevado. No fue sino hasta la mañana del día siguiente,
al leer su email, que se enteró que, durante su ausencia en el banco, había
recibido una visita del supermercado Mes\%, al cual había justamente encargado
una compra el día anterior. Utilizando un vínculo provisto dentro del mismo
mensaje, programó la visita para ese mismo día, al mediodía, y luego se ató un
piolín en el dedo corazón para recordarlo. Entonces miró un poco de televisión,
y tras terminar el programa, cuando se dispuso a cambiar de canal, se dio cuenta
que su dedo, el del piolín obviamente, estaba totalmente ennegrecido y arrugado.
Peor aún, a pesar de desatarlo, este había perdido la sensibilidad, y no
recuperaba su rosadito color habitual. \textbf{\emph{Frank}} se asustó tanto,
que corrió raudo hasta la calle, y tomó un taxi hasta el hospital más cercano,
sin advertir que estaba en pijama, y que este no dejaba nada a la imaginación.
En el hospital, les explicó que, por razones que no podemos repetir, para él era
muy importante este dedo, y que no podía perderlo. Entonces le dieron una bata
para que pueda poner sobre el pijama y proteger la sensibilidad del resto de los
pacientes, y le realizaron estrambóticos procedimientos médicos. Luego de un par
de horas \textbf{\emph{Frank}} pudo recuperar el funcionamiento habitual de su
dedo. Cuando el médico le preguntó que por qué se le había puesto así el dedo,
\textbf{\emph{Frank}} le explicó que se había atado algo. Tras lo cual el médico
hizo una obvia segunda pregunta, lo cual provocó algo similar a un click en
algún recóndito lugar del cerebro de \textbf{\emph{Frank}}, seguido de una
catarata de imágenes mentales, la mayoría de ellas relacionadas al supermercado
Mes\%. Entonces, repentinamente, se levantó, corrió hasta la puerta del
hospital, y tomó un taxi nuevamente hasta su casa. Al llegar, abrió su mail,
para enterarse de que nuevamente había perdido la entrega. Se le informaba,
además, de que su usuario ``\texttt{f\_estein}'' había perdido la posibilidad de
realizar compras contrareembolso, hasta pagar una multa de \$100, lo cual
ciertamente lo puso de muy mal humor.

\subsubsection{G: Cancelación de pedido}

\textbf{\emph{Gabriel}} había hecho un pedido el lunes a la mañana, pagándolo con tarjeta de crédito. A la tarde
descubre mejores precios en los chinos de la vuelta, y decide cancelar la compra
para hacerla en ese lugar que le resulta más económico (haciendo las cuentas,
llega a la conclusión de que con la diferencia podía comprar un videojuego para
sus hijos). Entonces, vuelve rápido de los chinos a su casa, y entusiasmado
chequea que aún pueda hacer la cancelación. Para ello ingresa al sitio
\texttt{www.mesporciento.com} con su cuenta ``\texttt{gabi\_25x8}'', poniendo su
contraseña, y luego seleccionando el pedido vigente. Selecciona la opción de
modificación de pedido y posteriormente, al comprobar que no está cerrado, lo
cancela.

El sistema le confirma la cancelación y se le hace una nota de crédito para
reintegrarle el dinero. Finalmente  \textbf{\emph{Gabriel}} vuelve contento a
los chinos a hacer la compra y con lo que le sobra, compra el videojuego para
sus hijos (¡aunque como padre responsable, primero lo juega el para enseñarle a
ellos!).

\subsubsection{H: Modificación de pedido.}

En la casa de \textbf{\emph{Helena}} van a festejar la navidad. Unas horas
después de que hacer un pedido con la mercadería necesaria para preparar la
cena, su esposo \textbf{\emph{Horacio}} le comenta que habían sido invitados a
la cena su vecina \textbf{\emph{Hermenegilda}}, junto con sus hijos, debido a
que el marido y los parientes de esta aún no habían regresado del exterior, ya
que se encontraban en un viaje por asuntos laborales. Por este motivo,
\textbf{\emph{Helena}} decide modificar el pedido que tenía hecho, incrementando
la cantidad de productos comprados (como por ejemplo las gaseosas). Para ello,
ingresa al sitio \texttt{www.mesporciento.com} validando los datos de su cuenta,
y chequea que aún no haya sido cerrado dicho pedido. Al comprobarlo, selecciona
la opción para modificarlo, y agrega la mercadería que desea. Luego selecciona
contrareembolso como opción de pago, y sistema le confirma la modificación. Dado
que el pedido había sido originalmente pagado de forma online,
\textbf{\emph{Helena}} deberá abonar la diferencia en efectivo en cuanto lo
reciba.


\subsection{I: Preparar y cerrar pedido}

\textbf{\emph{Ismael}}, un empleado del depósito, entra al sistema interno y
revisa los pedidos pendientes. Selecciona el primero de la lista (están
ordenados por prioridad) y lo prepara poniendo en una caja los productos
encargados. Luego lo cierra en el sistema interno.


\subsection{J: Alarma y reposición por bajo stock en depósitos}

Luego de que una clienta comprara diez packs de gaseosas Coca-Cola, se agotó el
stock de dicho producto, por lo que saltó  la alarma del sistema. El sistema
entonces le avisa al Departamento de Stock de forma automática. Un superior del
Departamento de Stock recibe el aviso y hace los pedidos correspondientes a
Logística, quien finalmente se encarga de  reponer el stock.
\fixme[arreglar]

%\subsubsection{Administración inteligente de productos de depósitos}
%\fixme[El administrador ingresa al sistema, ve estadísticas,
%modifica productos vigentes (abm)]

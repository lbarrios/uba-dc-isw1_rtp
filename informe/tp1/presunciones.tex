%--
%  Presunciones: acá deberían listar aquellas cuestiones que asumieron por
%  encima del enunciado. Estas cuestiones pueden provenir de alguna consulta con
%  docentes. También pueden provenir de alguna especulación o interpretación que
%  el grupo hizo.
%--

\section{Presunciones / Aclaraciones adicionales}

Se incluyen a continuación aquellas presunciones que se utilizaron como
complemento a las indicaciones originales. Estas pueden ser, o bien consultadas
al \textbf{CEO} en caso de considerarlas lo suficientemente relevantes, o
asumidas/interpretadas por cuenta propia.

\begin{itemize}

  \item En caso de que la cadena de supermercados decida inaugurar más
\textbf{\red{depósitos}}, tanto la reposición de stock como los traslados
interdepósito estarán a cargo de la compañía, sin ningún tipo de intervención
por parte del \textbf{\blue{sistema}}. A efectos prácticos, entonces, se asumirá
que existe un solo \textbf{\red{depósito}}, cuyo stock es la sumatoria del stock
de todos los \textbf{\red{depósitos}}, y que todos los pedidos serán entregados
desde el mismo.

  \item Una empresa de \textbf{\magenta{logística}} se encargará de todos los
traslados que se originen en cualquier \textbf{\red{depósito}}. El
\textbf{\blue{sistema}} deberá contar con una interfaz adecuada que actúe como
canal de comunicación bilateral con dicha empresa. Este canal deberá ser apto
tanto para realizar consultas y pedidos de forma online, a través de solicitudes
directas al servidor de la compañía de \textbf{\magenta{logística}}, como para
informar o ingresar datos de forma indirecta, a través de la impresión o la
carga manual de: remitos de traslado, comprobantes de pago, facturas, hojas de
ruta, etcétera.

  \item El \textbf{\blue{sistema}} deberá llevar, en todo momento, un control
activo (en tiempo real, en línea) del stock del \textbf{\red{depósito}}.

  \item Existe un \textbf{\orange{Departamento de Stock}}, que se encarga de
organizar los pedidos a los \textbf{\yellow{proveedores}} cuando un \textbf{\red{depósito}} se
queda sin stock. Todo esto se hace por fuera del sistema.

  \item Cada pedido se cierra al momento en que el \textbf{\red{depósito}}
informa que el mismo fue armado. Consecuentemente, se asume que estos no
necesariamente permanecerán abiertos hasta el momento en que son retirados por
\textbf{\magenta{logística}}.

  \item \textbf{\green{sucursal}}, \textbf{\red{depósito}},
  \textbf{\magenta{logística}} y \textbf{\orange{Departamento de Stock}} podrán acceder
    al sistema por medio de una terminal pre-autenticada.

\end{itemize}
